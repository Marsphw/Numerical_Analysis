\documentclass[a4paper]{article}
\usepackage[affil-it]{authblk}
\usepackage[backend=bibtex,style=numeric]{biblatex}
\usepackage{amsmath, amsthm, amssymb, amsfonts}
\usepackage{titlesec}
\usepackage{graphicx}
\usepackage{multirow}
\usepackage{hyperref}
\usepackage{cleveref}
\hypersetup{
    colorlinks=true,    % 彩色链接
    linkcolor=blue,     % 内部链接的颜色
    citecolor=red,      % 引用的颜色
    urlcolor=cyan       % 外部链接的颜色
}
\usepackage{geometry}
\usepackage[lined, ruled]{algorithm2e}
\usepackage{listings}
\lstset{
    basicstyle = \ttfamily,
    keywordstyle = \color{blue},
    commentstyle = \color{green!50!black}\itshape,
    stringstyle = \color{orange},
    numbers = left,
    numberstyle = \scriptsize,
    numbersep = 5pt,
    breaklines = true,
}
\geometry{margin=1.5cm, vmargin={0pt,1cm}}
\setlength{\topmargin}{-1cm}
\setlength{\paperheight}{29.7cm}
\setlength{\textheight}{25.3cm}
\titleformat{\section}{\Large\bfseries}{\Roman{section}}{1em}{}
\titleformat{\subsection}{\large\bfseries}{\Roman{section}.\alph{subsection}}{1em}{}
\titleformat{\subsubsection}{\normalsize\bfseries}{\Roman{section}.\alph{subsection}.\roman{subsubsection}}{1em}{}

\crefname{algorithm}{Algorithm}{Algorithms}
\Crefname{algorithm}{Algorithm}{Algorithms}
\crefname{equation}{Equation}{Equations}
\Crefname{equation}{Equation}{Equations}
\crefname{figure}{Figure}{Figures}
\Crefname{figure}{Figure}{Figures}
\crefname{table}{Table}{Tables}
\Crefname{table}{Table}{Tables}

\crefalias{algocf}{algorithm}

\renewcommand\arraystretch{1.2}

%\addbibresource{citation.bib}

\begin{document}
% =================================================
\title{\textbf{Numerical Analysis theoretical homework \# 5}}

\author{Peng Haowei 3220104816
  \thanks{E-mail address: \texttt{3220104816@zju.edu.cn}}}
\affil{(Information and Computing Science), Zhejiang University} 

\date{Due time: \today}

\maketitle

\section{Proof of Theorem 5.7}

In this section, we consider a inner product as 
\begin{equation}
    \langle u, v\rangle := \int_a^b \rho(t) u(t) \overline{v(t)}\ \mathrm{d}t,
    \label{eq:1_inner_product}
\end{equation}
where $\bar{v(t)}$ is the complex conjugate of $v(t)$ and the \textit{weight function} $\rho(x) \in \mathcal[a, b]$ satisfies $\rho(x) > 0$ for all $x \in (a, b)$.

And the 2-norm is defined as
\begin{equation}
    \|u\|_2 := \left(\int_a^b \rho(t) |u(t)|^2\ \mathrm{d}t\right)^{\frac{1}{2}}.
    \label{eq:1_2_norm}
\end{equation}

\begin{proof}
Above all, we need to prove the set $\mathcal{C}[a, b]$ of continuous function over $[a, b]$ is a vector space over $\mathbb{C}$.

\begin{enumerate}
    \item commutativity: for any $f, g \in \mathcal{C}[a, b]$, we have $f + g = g + f$ obviously.
    \item associativity: for any $f, g, h \in \mathcal{C}[a, b]$, we have $(f + g) + h = f + (g + h)$.
    \item compatibility: for any $f \in \mathcal{C}[a, b]$ and $m, n \in \mathbb{C}$, we have $(mn)f = m(nf)$.
    \item additive identity: for any $f \in \mathcal{C}[a, b]$, we have $0 + f = f$.
    \item additive inverse: for any $f \in \mathcal{C}[a, b]$, we have $g(x) = -f(x)$ for all $x \in [a, b]$ to satisfy $f + g = 0$.
    \item multiplicative identity: for any $f \in \mathcal{C}[a, b]$, we have $1f = f$.
    \item distributive laws: for any $f, g \in \mathcal{C}[a, b]$ and $m, n \in \mathbb{C}$, we have $m(f + g) = mf + mg$ and $(m + n)f = mf + nf$.
\end{enumerate}

Consequently, the set $\mathcal{C}[a, b]$ is a vector space over $\mathbb{C}$.
Then for the proof of inner product space, it follows

\begin{enumerate}
    \item real positivity: for any $f \in \mathcal{C}[a, b]$, we have $\langle f, f\rangle = \int_a^b \rho(t) f(t) \overline{f(t)} \mathrm{d}t = \int_a^b \rho(t) (f(t))^2 \mathrm{d}t \geqslant 0$.
    \item definiteness: from $\langle f, f \rangle = \int_a^b \rho(t) (f(t))^2 \mathrm{d}t$, we have $\langle f, f \rangle = 0$ iff $f = 0$.
    \item additivity in the first slot: for any $f, g, h \in \mathcal{C}[a, b]$, we have $\langle f + g, h \rangle = \int_a^b \rho(t) (f + g)(t) \overline{h(t)} \mathrm{d}t = \int_a^b \rho(t) (f(t) + g(t)) \overline{h(t)} \mathrm{d}t = \int_a^b \rho(t) f(t) \overline{h(t)} \mathrm{d}t + \int_a^b \rho(t) g(t) \overline{h(t)} \mathrm{d}t = \langle f, h \rangle + \langle g, h \rangle$.
    \item homogeneity in the first slot: for any $f, g \in \mathcal{C}[a, b]$ and $m \in \mathbb{C}$, we have $\langle mf, g \rangle = \int_a^b \rho(t) (mf)(t) \overline{g(t)} \mathrm{d}t = m\int_a^b \rho(t) f(t) \overline{g(t)} \mathrm{d}t = m\langle f, g \rangle$.
    \item conjugate symmetry: for any $f, g \in \mathcal{C}[a, b]$, we have $\langle f, g \rangle = \int_a^b \rho(t) f(t) \overline{g(t)} \mathrm{d}t = \int_a^b \overline{\rho(t)} \overline{g(t)} \overline{(\overline{f(t)})} \mathrm{d}t = \overline{\int_a^b \rho(t) g(t) \overline{f(t)} \mathrm{d}t} = \overline{\langle g, f \rangle}$.
\end{enumerate}
Therefore, the set $\mathcal{C}[a, b]$ is an inner-product space over $\mathbb{C}$ with its inner product as \cref{eq:1_inner_product}.

According to the definition, we have
\begin{equation}
    \|u\|_2 =  \sqrt{\langle u, u \rangle} = \left(\int_a^b \rho(t) |u(t)|^2 \mathrm{d}t\right)^{\frac{1}{2}}.
    \label{eq:1_2_norm_2}
\end{equation}

\end{proof}

\section{Problems of the Chebyshev polynomials}

\subsection{Show the orthogonality}

The Chebyshev polynomials of the first kind are defined as
\begin{equation}
    T_n(x) = \cos (n \arccos x).
    \label{eq:2_chebyshev_polynomials}
\end{equation}

\begin{proof}
    Denote $x = \cos \theta$, then we have
    \begin{equation}
        \mathrm{d}x = -\sin \theta\ \mathrm{d} \theta = -\frac{1}{\sqrt{1 - x^2}}\ \mathrm{d}\theta.
        \label{eq:2_chebyshev_polynomials_derivative}
    \end{equation}

    For all $m, n \in \mathbb{N},\ m \ne n$, it follows
    \begin{equation}
        \begin{aligned}
            \int_{-1}^1 T_m(x) T_n(x) \frac{1}{\sqrt{1 - x^2}} \mathrm{d}x &= \int_0^{\pi} \cos (m\theta) \cos (n\theta)\ \mathrm{d} \theta \\
            &= \frac{1}{2} \int_0^{\pi} [\cos ((m + n)\theta) - \cos ((m - n)\theta)]\ \mathrm{d} \theta \\
            &= \frac{\sin((m + n)\theta)}{m + n} \Big|_0^{\pi} - \frac{\sin((m - n)\theta)} {m - n} \Big|_0^{\pi} \\
            &= 0,
        \end{aligned}
        \label{eq:2_chebyshev_polynomials_orthogonality}
    \end{equation}
    which yields the orthogonality of the Chebyshev polynomials.
\end{proof}

\subsection{Normalize some polynomials to get an orthonormal system}

In this problem, we need to normalize the first three Chebyshev polynomials of the first kind to get an orthonormal system.

We have 
\begin{equation}
    \begin{aligned}
        \langle T_0, T_0 \rangle &= \int_0^{\pi} 1\ \mathrm{d}\theta = \pi \\
        \langle T_1, T_1 \rangle &= \int_0^{\pi} \cos^2 \theta\ \mathrm{d}\theta = \frac{\pi}{2} \\
        \langle T_2, T_2 \rangle &= \int_0^{\pi} \cos^2 2\theta\ \mathrm{d}\theta = \frac{\pi}{2},
    \end{aligned}
    \label{eq:2_chebyshev_polynomials_norm}
\end{equation}
hence the result of normalization is 
\begin{equation}
    u_1^* = \frac{1}{\sqrt{\pi}},\ u_2^* = \sqrt{\frac{2}{\pi}}x,\ u_3^* = \sqrt{\frac{2}{\pi}}(2x^2 - 1).
    \label{eq:2_chebyshev_polynomials_normalized}
\end{equation}

\section{Least-square approximation of a continuous function}

In this section, we need to approximate the circular arc given by the equation
\begin{equation}
    y(x) = \sqrt{1 - x^2},\quad x \in [-1, 1]
    \label{eq:3_circular_arc}
\end{equation}
by a quadratic polynomial.

\subsection{Using Fourier expansion}

As the weight function is $\rho(x) = \frac{1}{\sqrt{1 - x^2}}$, we use the Chebyshev polynomials of the first kind in \cref{eq:2_chebyshev_polynomials_normalized}, and we have the Fourier coefficient of $y(x)$ as 
\begin{equation}
    \begin{aligned}
        b_1 &= \int_{-1}^1 \frac{1}{\sqrt{1 - x^2}} \frac{1}{\sqrt{\pi}} \sqrt{1 - x^2}\ \mathrm{d} x = \frac{2}{\sqrt{\pi}}, \\
        b_2 &= \int_{-1}^1 \frac{1}{\sqrt{1 - x^2}} \sqrt{\frac{2}{\pi}}x \sqrt{1 - x^2}\ \mathrm{d} x = 0, \\
        b_3 &= \int_{-1}^1 \frac{1}{\sqrt{1 - x^2}} \sqrt{\frac{2}{\pi}}(2x^2 - 1) \sqrt{1 - x^2} \ \mathrm{d} x = -\frac{2}{3}\sqrt{\frac{2}{\pi}}.
    \end{aligned}
    \label{eq:3_fourier_coefficient}
\end{equation}

So the minimizing polynomial is 
\begin{equation}
    \hat{\varphi} = \frac{2}{\pi} - \frac{4}{3\pi}(2x^2 - 1).
    \label{eq:3_minimizing_polynomial}
\end{equation}

\subsection{Using normal equations}

Assuming that the best approximation of $y(x)$ is $\hat{\varphi} = a_0 + a_1 x + a_2 x^2$. To find it, we first construct the Gram matrix with $\rho = \frac{1}{\sqrt{1 - x^2}}$:
\begin{equation}
    G(1, x, x^2) = 
    \begin{bmatrix}
        \langle 1, 1 \rangle & \langle 1, x \rangle & \langle 1, x^2 \rangle \\
        \langle x, 1 \rangle & \langle x, x \rangle & \langle x, x^2 \rangle \\
        \langle x^2, 1 \rangle & \langle x^2, x \rangle & \langle x^2, x^2 \rangle \\
    \end{bmatrix}
    = \begin{bmatrix}
        \pi & 0 & \pi/2 \\
        0 & \pi/2 & 0 \\
        \pi/2 & 0 & \pi/2 \\
    \end{bmatrix}.
    \label{eq:3_gram_matrix}
\end{equation}

Then we calculate the vector 
\begin{equation}
    \mathbf{c} = \begin{bmatrix}
        \langle y(x), 1 \rangle \\
        \langle y(x), x \rangle \\
        \langle y(x), x^2 \rangle \\
    \end{bmatrix}
    = \begin{bmatrix}
        2 \\ 0 \\ 2/3 \\
    \end{bmatrix}.
    \label{eq:3_vector_c}
\end{equation}

The normal equations then yield 
\begin{equation}
    a_0 = \frac{8}{3\pi},\ a_1 = 0,\ a_2 = -\frac{4}{3\pi},
    \label{eq:3_normal_equations}
\end{equation}
which means the result of approximation is 
\begin{equation}
    \hat{\varphi} = \frac{8}{3\pi} - \frac{4}{3\pi}x^2.
    \label{eq:3_least_square_approximation}
\end{equation}

\section{Discrete least square via orthonormal polynomials}

In this section, we consider the table of sales record as follows:
\begin{table}[htbp]
    \centering
    \begin{tabular}{c|cccccccccccc}
        \hline
        $x$ & 1 & 2 & 3 & 4 & 5 & 6 & 7 & 8 & 9 & 10 & 11 & 12\\ \hline
        $y$ & 256 & 201 & 159 & 61 & 77 & 40 & 17 & 25 & 103 & 156 & 222 & 345 \\ \hline
    \end{tabular}
    \caption{Sales record}
    \label{tab:sales_record}
\end{table}

\subsection{Construct the orthonormal system}
\label{sec:construct_orthonormal_system}

In this section, we use 
\begin{equation}
    \langle u(t), v(t) \rangle = \sum_{i = 1}^N \rho(t_i) u(t_i) v(t_i)
    \label{eq:4_inner_product}
\end{equation}
as the inner product with $N = 12$ and $\rho(x) = 1$.

We have the independent list 
\begin{equation}
    u_1 = 1,\ u_2 = x,\ u_3 = x^2,
    \label{eq:4_independent_list}
\end{equation}
and firstly we get 
\begin{equation}
    v_1 = u_1 = 1,\ u_1^* = v_1 / \|v_1\| = \frac{1}{\sqrt{12}} = \frac{\sqrt{3}}{6}.
    \label{eq:4_first_column}
\end{equation}

According to the Gram-Schmidt process, it follows
\begin{equation}
    \begin{aligned}
        v_2 &= u_2 - \langle u_2, v_1^* \rangle v_1^* = x - (\frac{\sqrt{3}}{6})^2 \sum_{i = 1}^N x_i = x - \frac{13}{2}, \\
        u_2^* &= \frac{v_2}{\|v_2\|} = \frac{x - \frac{13}{2}}{\sqrt{143}}.
    \end{aligned}
    \label{eq:4_second_column}
\end{equation}

\begin{equation}
    \begin{aligned}
        v_3 &= u_3 - \langle u_3, v_1^* \rangle v_1^* - \langle u_3, v_2^* \rangle v_2^* \\
        &= x^2 - (\frac{\sqrt{3}}{6})^2 \sum_{i = 1}^N (x_i)^2 - \frac{x - \frac{13}{2}}{\sqrt{143}} \sum_{i = 1}^N x_i^2 \cdot \frac{x_i - \frac{13}{2}}{\sqrt{143}} \\
        &= x^2 - 13x + \frac{91}{3}, \\
        u_3^* &= \frac{v_3}{\|v_3\|} = \frac{v_3}{\frac{2}{3\sqrt{3003}}} = \frac{\sqrt{3003}}{2002} (x^2 - 13x + \frac{91}{3}).
    \end{aligned}
    \label{eq:4_third_column}
\end{equation}

\subsection{Find the best approximation}
\label{sec:find_best_approximation}

To find the best approximation $\hat{\varphi} = \sum_{i = 0}^2 a_i x^i$ such that $\|y - \hat{\varphi}\| \leqslant \|y - \sum_{i = 0}^2 b_i x^i$ for all $b_i \in \mathbb{R}$, we need to project the target function onto the orthonormal system we get in \cref{sec:construct_orthonormal_system}.
\begin{equation}
    \begin{aligned}
        \langle y, u_1^* \rangle &= \frac{\sqrt{3}}{6} \sum_{i = 1}^N y_i = \frac{\sqrt{3}}{6} \times 1662 = 277\sqrt{3} \approx 479.77807370, \\
        \langle y, u_2^* \rangle &= \frac{1}{\sqrt{143}} \sum_{i = 1}^N x_iy_i - \frac{\sqrt{143}}{22} \sum_{i = 1}^N y_i = \frac{1}{\sqrt{143}} \times 11392 + \frac{\sqrt{143}}{22} \times 1662 \approx 49.25465439, \\
        \langle y, u_3^* \rangle &= \frac{\sqrt{3003}}{2002} (\sum_{i = 1}^N x_i^2y_i + 13\sum_{i = 1}^N x_iy_i + \frac{91}{3}\sum_{i = 1}^N y_i) = \frac{\sqrt{3003}}{2002}(109750 - 13 \times 11392 + \frac{91}{3} \times 1662) \approx 330.33066714.
    \end{aligned}
    \label{eq:4_projection}
\end{equation}

Therefore, the best approximation $\hat{\varphi}$ is 
\begin{equation}
    \hat{\varphi} = \sum_{i = 1}^3 \langle y, u_i^* \rangle u_i^* = 9.0420 x^2 - 113.4266 x + 386.0000,
    \label{eq:4_best_approximation}
\end{equation}
which is the same as the result on the Example 5.55 in the textbook.

\subsection{Consider the reuse of the values}

We consider the reuse of the values in our calculation with the assumption that values of $N$ and $x_i$'s are the same while the the values of $y_i$'s change.

Viewing the calculation in \cref{sec:construct_orthonormal_system,sec:find_best_approximation}, we find that orthonormal system will not change as the values of $y_i$'s are different, meaning that they can be reused. For the coefficients $a_0, a_1, a_2$ of $\hat{\varphi}$, they cannot be used.

By using orthonormal polynomials, we can reuse the values of bases and calculate te results with some easy arithmetic operations. When we use normal equations, we have to make a matrix calculation of $G^{-1} \mathbf{c}$.

\section{Proof of the Theorem 5.66 and Lemma 5.67}

First we need to define the \textit{Moore-Penrose inverse}. The \textit{Moore-Penrose inverse} or \textit{pseudoinverse} or \textit{generalized inverse} of a matrix $A \in \mathbb{F}^{m \times n}$ where $\mathbb{F} = \mathbb{R}, \mathbb{C}$ is the matrix $A^+ \in \mathbb{F}^{m \times n}$ given by 
\begin{equation}
    A^{\dagger} = V \begin{pmatrix} \Sigma_r^{-1} & O \\ O & O \end{pmatrix} U^T,
    \label{eq:5_pseudoinverse}
\end{equation}
where $r$ is the rank of $A$ and it follows
\begin{equation}
    A = U \Sigma V^T,
    \label{eq:5_svd}
\end{equation}
which is the singular value decomposition of $A$.

\subsection{Proof of Theorem 5.66}

For the first property in the theorem, we have
\begin{equation}
    \begin{aligned}
        A A^{\dagger} A &= AV \begin{pmatrix} \Sigma_r^{-1} & O \\ O & O \end{pmatrix} U^T A \\
        &= U \Sigma_r V^T V \begin{pmatrix} \Sigma_r^{-1} & O \\ O & O \end{pmatrix} U^T U \Sigma_r V^T \\
        &= U \Sigma_r V^T \\
        &= A.
    \end{aligned}
    \label{eq:5_property1}
\end{equation}

The second property in the theorem is proved as follows.
\begin{equation}
    \begin{aligned}
        A^{\dagger} A A^{\dagger} &= V \begin{pmatrix} \Sigma_r^{-1} & O \\ O & O \end{pmatrix} U^T A V \begin{pmatrix} \Sigma_r^{-1} & O \\ O & O \end{pmatrix} U^T \\
        &= V \begin{pmatrix} \Sigma_r^{-1} & O \\ O & O \end{pmatrix} U^T U \Sigma_r V^T V \begin{pmatrix} \Sigma_r^{-1} & O \\ O & O \end{pmatrix} U^T \\
        &= V \begin{pmatrix} \Sigma_r^{-1} & O \\ O & O \end{pmatrix} U^T  \\
        &= A^{\dagger}.
    \end{aligned}
    \label{eq:5_property2}
\end{equation}

As for the Hermitian property, we have
\begin{equation}
    \begin{aligned}
        (AA^{\dagger})^* &= (U \Sigma_r V^T V \begin{pmatrix} \Sigma_r^{-1} & O \\ O & O \end{pmatrix} U^T)^* \\
        &= (U \begin{pmatrix} I_n & O \\ O & O \end{pmatrix} U^T)^* \\
        &= (U^T)^* \begin{pmatrix} I_n & O \\ O & O \end{pmatrix} U^* \\
        &= U \begin{pmatrix} I_n & O \\ O & O \end{pmatrix} U^T \\
        &= AA^{\dagger}.
    \end{aligned}
    \label{eq:5_hermitian}
\end{equation}
Similarly we can proof $A^{\dagger}A$ is Hermitian.

\section*{ \center{\normalsize {Acknowledgement}} }

In the process of writing this report, I use \href{https://kimi.moonshot.cn/}{\textit{Kimi AI}} to help me translate something and write this report by \TeX.

\end{document}