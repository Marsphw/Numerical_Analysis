\documentclass[a4paper]{article}
\usepackage[affil-it]{authblk}
\usepackage[backend=bibtex,style=numeric]{biblatex}
\usepackage{amsmath, amsthm, amssymb, amsfonts}
\usepackage{titlesec}
\usepackage{graphicx}
\usepackage{multirow}
\usepackage{hyperref}
\usepackage{cleveref}
\hypersetup{
    colorlinks=true,    % 彩色链接
    linkcolor=blue,     % 内部链接的颜色
    citecolor=red,      % 引用的颜色
    urlcolor=cyan       % 外部链接的颜色
}
\usepackage{geometry}
\usepackage[lined, ruled]{algorithm2e}
\usepackage{listings}
\lstset{
    basicstyle = \ttfamily,
    keywordstyle = \color{blue},
    commentstyle = \color{green!50!black}\itshape,
    stringstyle = \color{orange},
    numbers = left,
    numberstyle = \scriptsize,
    numbersep = 5pt,
    breaklines = true,
}
\geometry{margin=1.5cm, vmargin={0pt,1cm}}
\setlength{\topmargin}{-1cm}
\setlength{\paperheight}{29.7cm}
\setlength{\textheight}{25.3cm}
%\titleformat{\section}{\Large\bfseries}{\Roman{section}}{1em}{}
%\titleformat{\subsection}{\large\bfseries}{\Roman{section}.\alph{subsection}}{1em}{}
%\titleformat{\subsubsection}{\normalsize\bfseries}{\Roman{section}.\alph{subsection}.\arabic{subsubsection}}{1em}{}

\crefname{algorithm}{Algorithm}{Algorithms}
\Crefname{algorithm}{Algorithm}{Algorithms}
\crefname{equation}{Equation}{Equations}
\Crefname{equation}{Equation}{Equations}
\crefname{figure}{Figure}{Figures}
\Crefname{figure}{Figure}{Figures}
\crefname{table}{Table}{Tables}
\Crefname{table}{Table}{Tables}

\crefalias{algocf}{algorithm}

\theoremstyle{definition}
\newtheorem{lemma}{Lemma}
\newtheorem{definition}{Definition}
\newtheorem{theorem}{Theorem}

\renewcommand\arraystretch{1.2}

%\addbibresource{citation.bib}

\begin{document}
% =================================================
\title{\textbf{Design Document of the Project}}

\author{Peng Haowei 3220104816
  \thanks{E-mail address: \texttt{3220104816@zju.edu.cn}}}
\affil{(Information and Computing Science), Zhejiang University} 

\date{Due time: \today}

\maketitle

\begin{abstract}
  This project concentrates on the development of a package for piecewise-polynomial splines(pp-Form) and B-splines(B-Form). 

  Splines are a kind of mathematical function passing through a set of data points that can be used to approximate a smooth curve or a function, with each segment between two points being a polynomial and the entire curve is continuous. 
  The core of the construction of a spline is to solve a linear system of equations with the coefficient matrix is tridiagonal. 
  \end{abstract}

\section{File Structure}

In this project, the file structure is shown as follows:
\begin{itemize}
  \item \verb|src| -- the source code of the project, including the main program and the library files.
  \item \verb|doc| -- the documentation of the project, including the design document, the report of the project.
  \item \verb|figure| -- the figures used in the report.
  \item \verb|data| -- the data used in the project, including the sample data and the test data.
  \item \verb|README.md| -- the README file of the project.
  \item \verb|Makefile| -- the makefile used to compile, run the project and generate the documents.
\end{itemize}

\section{Splines}

\subsection{Introduction}

Splines are a special kind of mathematical functions that can be used to approximate a smooth curve or a function, with each segment between two points being a polynomial and the entire curve is continuous, which is a type of interpolation polynomials. 

\begin{definition}
  Given nonnegative integers $n, k$, and a strictly increasing sequence $(x_i)_{i = 1}^N$ that partitions the interval $[a, b]$, 
  \begin{equation}
    a = x_1 < x_2 < \cdots < x_N = b,
    \label{eq:partition}
  \end{equation}
  the set of \textit{spline functions of degree $n$ and smoothness class $k$} relative to the partitioned interval is 
  \begin{equation}
    \mathbb{S}_n^k = \{s: s \in \mathcal{C}^k[a, b];\ \forall i \in [1, N - 1],\ s\big|_{[x_i, x_{i + 1}]} \in \mathbb{P}_n\}.
    \label{eq:spline-set}
  \end{equation}
  The $x_i$'s are called \textit{knots} of the spline.
\end{definition}

The most commonly used spline functions are the cubic splines, i.e. $\mathbb{S}_3^2$.

\subsection{Boundary conditions of cubic splines}

We consider three kinds of boundary conditions:
\begin{itemize}
  \item A \textit{complete cubic spline} $s \in \mathbb{S}_3^2$ satisfies boundary conditions $s'(f; a) = f'(a)$ and $s'(f; b) = f'(b)$.
  \item A \textit{natural cubic spline} $s \in \mathbb{S}_3^2$ satisfies boundary conditions $s''(f; a) = 0$ and $s''(f; b) = 0$.
  \item A \textit{periodic cubic spline} $s \in \mathbb{S}_3^2$ is obtained from replacing $s(f; b) = f(b)$ with $s(f; b) = s(f; a),\ s'(f; b) = s'(f; a)$ and $s''(f; b) = s''(f; a)$.
\end{itemize}

\subsection{PP-form Splines}
\label{sec:pp-form-splines}

\begin{theorem}
  Denote $m_i = s'(f; x_i)$ for $s \in \mathbb{S}_3^2$. Then, for each $i = 2, 3, \cdots, N - 1$, we have 
  \begin{equation}
    \lambda_i m_{i - 1} + 2m_i + \mu_i m_{i + 1} = 3\mu_i f[x_i, x_{i + 1}] + 3\lambda_i f[x_{i - 1}, x_i],
    \label{eq:pp-form-spline-tridiagonal}
  \end{equation}
  where 
  \begin{equation}
    \mu_i = \frac{x_i - x_{i - 1}}{x_{i + 1} - x_{i - 1}}, \quad \lambda_i = \frac{x_{i + 1} - x_i}{x_{i + 1} - x_{i - 1}}.
    \label{eq:pp-form-spline-coefficients}
  \end{equation}
\end{theorem}

According to the above lemma, we can construct the cubic splines in $\mathbb{S}_3^2$ by solving the linear system of equations with the coefficient matrix is tridiagonal. 
More specifically, the polynomials $p_i$ are given by 
\begin{equation}
  p_i(x) = f_i + (x - x_i)m_i + \frac{3K_i - 2m_i - m_{i + 1}}{x_{i + 1} - x_i} (x - x_i)^2 + \frac{m_i + m_{i + 1} - 2K_i}{(x_{i + 1} - x_i)^2} (x - x_i)^3,
  \label{eq:pp-form-spline-polynomial}
\end{equation}
where $K_i = f[x_i, x_{i + 1}]$.

The second derivative is given by 
\begin{equation}
  p''_{i}(x) = 2\frac{3K_i - 2m_i - m_{i + 1}}{x_{i + 1} - x_i} + 6\frac{m_i + m_{i + 1} - 2K_i}{(x_{i + 1} - x_i)^2}(x - x_i).
  \label{eq:pp-form-spline-second-derivative}
\end{equation}

So the different boundary conditions can be transferred into different linear systems, which are illustrated in the following sections.

\subsubsection{Complete cubic spline}

The complete cubic spline can be expressed as
\begin{equation}
  \begin{bmatrix}
    1 &  &  &  &  &  &  &  & \\
    \lambda_2 & 2 & \mu_2 &  &  &  &  &  & \\
    & \lambda_3 & 2 & \mu_3 &  &  &  &  & \\
    &   &   & \ddots &  &  &  &  & \\
    &   &   & \lambda_i & 2 & \mu_i &  &  & \\
    &   &   &   &   & \ddots &  &  & \\
    &   &   &   &   & \lambda_{N - 2} & 2 & \mu_{N - 2} & \\
    &   &   &   &   &   & \lambda_{N - 1} & 2 & \mu_{N - 1} \\
    &   &   &   &   &   &   &   & 1 \\
  \end{bmatrix}
  \begin{bmatrix}
    m_1 \\ m_2 \\ m_3 \\ \vdots \\ m_i \\ \vdots \\ m_{N - 2} \\ m_{N - 1} \\ m_N
  \end{bmatrix}
  = \begin{bmatrix}
    f'(a) \\
    3\mu_2 K_2 + 3\lambda_2 K_1 \\
    3\mu_3 K_3 + 3\lambda_3 K_2 \\
    \vdots \\
    3\mu_i K_i + 3\lambda_i K_{i - 1} \\
    \vdots \\
    3\mu_{N - 2} K_{N - 2} + 3\lambda_{N - 2} K_{N - 3} \\
    3\mu_{N - 1} K_{N - 1} + 3\lambda_{N - 1} K_{N - 2} \\
    f'(b)
    \end{bmatrix}.
  \label{eq:pp-form-spline-linear-system-complete}
\end{equation}

\subsubsection{Natural cubic spline}

And the natural cubic spline can be expressed as
\begin{equation}
  \begin{bmatrix}
    1 &  &  &  &  &  &  &  & \\
    \lambda_2 & 2 & \mu_2 &  &  &  &  &  & \\
    & \lambda_3 & 2 & \mu_3 &  &  &  &  & \\
    &   &   & \ddots &  &  &  &  & \\
    &   &   & \lambda_i & 2 & \mu_i &  &  & \\
    &   &   &   &   & \ddots &  &  & \\
    &   &   &   &   & \lambda_{N - 2} & 2 & \mu_{N - 2} & \\
    &   &   &   &   &   & \lambda_{N - 1} & 2 & \mu_{N - 1} \\
    &   &   &   &   &   &   &   & 1 \\
  \end{bmatrix}
  \begin{bmatrix}
    m_1 \\ m_2 \\ m_3 \\ \vdots \\ m_i \\ \vdots \\ m_{N - 2} \\ m_{N - 1} \\ m_N
  \end{bmatrix}
  = \begin{bmatrix}
    0 \\
    3\mu_2 K_2 + 3\lambda_2 K_1 \\
    3\mu_3 K_3 + 3\lambda_3 K_2 \\
    \vdots \\
    3\mu_i K_i + 3\lambda_i K_{i - 1} \\
    \vdots \\
    3\mu_{N - 2} K_{N - 2} + 3\lambda_{N - 2} K_{N - 3} \\
    3\mu_{N - 1} K_{N - 1} + 3\lambda_{N - 1} K_{N - 2} \\
    0
    \end{bmatrix}.
  \label{eq:pp-form-spline-linear-system-natural}
\end{equation}

\subsubsection{Periodic cubic spline}

As for the periodic cubic spline, we have $s(f; b) = s(f; a)$, $s'(f; b) = s'(f; a)$ and $s''(f; b) = s''(f; a)$, so we can translate $S_3(x)$ in $[x_1, x_2]$ to $[x_N, x_N + x_2 - x_1]$ to satisfy the conditions. Therefore, we have the following linear system.
\begin{equation}
  \begin{bmatrix}
    1 &  &  &  &  &  &  &  & -1 \\
    \lambda_2 & 2 & \mu_2 &  &  &  &  &  & \\
    & \lambda_3 & 2 & \mu_3 &  &  &  &  & \\
    &   &   & \ddots &  &  &  &  & \\
    &   &   & \lambda_i & 2 & \mu_i &  &  & \\
    &   &   &   &   & \ddots &  &  & \\
    &   &   &   &   & \lambda_{N - 2} & 2 & \mu_{N - 2} & \\
    &   &   &   &   &   & \lambda_{N - 1} & 2 & \mu_{N - 1} \\
    \mu_N &   &   &   &   &   &   & \lambda_N & 2 \\
  \end{bmatrix}
  \begin{bmatrix}
    m_1 \\ m_2 \\ m_3 \\ \vdots \\ m_i \\ \vdots \\ m_{N - 2} \\ m_{N - 1} \\ m_N
  \end{bmatrix}
  = \begin{bmatrix}
    0 \\
    3\mu_2 K_2 + 3\lambda_2 K_1 \\
    3\mu_3 K_3 + 3\lambda_3 K_2 \\
    \vdots \\
    3\mu_i K_i + 3\lambda_i K_{i - 1} \\
    \vdots \\
    3\mu_{N - 2} K_{N - 2} + 3\lambda_{N - 2} K_{N - 3} \\
    3\mu_{N - 1} K_{N - 1} + 3\lambda_{N - 1} K_{N - 2} \\
    3\mu_N K_1 + 3\lambda_N \frac{f(x_1) - f(x_{N - 1})}{x_N - x_{N - 1}}
    \end{bmatrix}
  \label{eq:pp-form-spline-linear-system-periodic}
\end{equation}
where 
\begin{equation}
    \mu_N = \frac{x_N - x_{N - 1}}{x_N - x_{N - 1} + x_2 - x_1},\quad \lambda_N = \frac{x_2 - x_1}{x_N - x_{N - 1} + x_2 - x_1}.
  \label{eq:pp-form-spline-linear-system-periodic-b}
\end{equation}

\subsection{B-form splines}

The B-splines are a special case of the PP-splines. From \cref{sec:pp-form-splines} we can know that we need to solve a linear system again when we change only one value of $f_i$. B-splines overcome this problem by using a series of truncated power basis functions.

\begin{definition}
  \textit{B-splines} are defined recursively by 
  \begin{equation}
    B_i^{n + 1}(x) = \frac{x - t_{i- 1}}{t_{i + n} - t_{i - 1}}B_i^n(x) + \frac{t_{i + n + 1} - x}{t_{i + n + 1} - t_i}B_{i + 1}^n(x).
    \label{eq:b-spline-recursive}
  \end{equation}
  The recursion base is the B-spline of degree zero,
  \begin{equation}
    B_1^0(x) = \begin{cases}
      1 & \text{if}\ x \in (t_{i - 1}, t_i], \\
      0 & \text{otherwise}.
    \end{cases}
  \end{equation}
\end{definition}

\begin{theorem}
  The following list of B-splines is a basis of $\mathbb{S}_n^{n - 1}(t_1, t_2, \cdots, t_N)$,
  \begin{equation}
    B_{2 - n}^n(x), B_{3 - n}^n(x), \cdots, B_{N}^n(x).
    \label{eq:b-spline-basis}
  \end{equation}
\end{theorem}

\begin{theorem}
  (Support of B-splines). For $n \in \mathbb{N}^+$, the interval of support of $B_i^n$ is $[t_{i - 1}, t_{i + n}]$ where 
  \begin{equation}
    \forall x \in (t_{i - 1}, t_{i + n}),\ B_i^n(x) > 0.
    \label{eq:b-spline-support}
  \end{equation}
  \label{thm:b-spline-support}
\end{theorem}

According to definition and theorems above, we can see that we need to add additional knots. Without loss of generality, we will assume 
\begin{equation}
  \begin{aligned}
    t_{1 - i} = t_1 - ih_1,& \quad \forall i \in \mathbb{N}, \\
    t_{N + i} = t_N + ih_2,& \quad \forall i \in \mathbb{N}.
  \end{aligned}
  \label{eq:cubic-b-spline-knots}
\end{equation}
where $h_1 = t_2 - t_1,\ h_2 = t_N - t_{N - 1}$, in \cref{sec:complete-cubic-b-splines,sec:natural-cubic-b-splines}.

\vspace{1em}
And we have a theorem about the derivatives of B-splines yielded by \cref{eq:b-spline-recursive}.
\begin{theorem}
  For $n \geqslant 2$, we have 
  \begin{equation}
    \frac{\mathrm{d}}{\mathrm{d}x}B_i^n(x) = \frac{nB_i^{n - 1}(x)}{t_{i + n + 1} - t_{i - 1}} - \frac{nB_{i + 1}^{n - 1}(x)}{t_{i + n} - t_i}, \quad \forall x \in \mathbb{R}.
    \label{eq:b-spline-derivative}
  \end{equation}
\end{theorem}

\subsubsection{Complete cubic B-splines}
\label{sec:complete-cubic-b-splines}

According to \cref{thm:b-spline-support}, we can assume the cubic B-spline is 
\begin{equation}
  S(x) = \sum_{i = -1}^N a_i B_i^3(x),
  \label{eq:complete-cubic-b-spline}
\end{equation}
where $a_i$ are the coefficients and 
\begin{equation}
  a_{-1} = \frac{3t_2 - 3t_1}{t_3 + t_2 - 2t_1} a_1 - 2(t_2 - t_1)f'(t_1),\quad a_N = \frac{3(t_N - t_{N - 1})}{2t_N - t_{N - 1} - t_{N - 2}} + 2(t_N - t_{N - 1})f'(t_N),
  \label{eq:complete-cubic-b-spline-coefficients}
\end{equation}
we can write the linear system for the coefficients $M\mathbf{a} = \mathbf{b}$ with
\begin{equation}
  \begin{aligned}
    \mathbf{b}^T &= [f(t_1) + \frac{(t_2 - t_1)f'(t_1)}{3}, f(t_2), \cdots, f(t_{N - 1}), f(t_N) - \frac{t_N - t_{N - 1}f'(t_N)}{3}], \\
    M &= \begin{bmatrix}
      \frac{1}{3} + \frac{t_3 - t_1}{2(t_3 + t_2 - 2t_1)} & \frac{t_2 - t_1}{t_3 + t_2 - 2t_1} &    &    &    \\
      B_0^3(t_2) & B_1^3(t_2) & B_2^3(t_2) &    &    \\
        & \ddots & \ddots & \ddots &    \\
        &    & B_{N - 3}^3(t_{N - 1}) & B_{N - 2}^3(t_{N - 1}) & B_{N - 1}^3(t_{N - 1}) \\
        &    &    & \frac{t_N - t_{N - 1}}{2t_N - t_{N - 1} - t_{N - 2}} & \frac{1}{3} + \frac{t_N - t_{N - 2}}{2(2t_N - t_{N - 1} - t_{N - 2})}
    \end{bmatrix},
  \end{aligned}
  \label{eq:complete-cubic-b-spline-linear-system}
\end{equation}
and $\mathbf{a}^T = [a_0, a_1, \cdots, a_{N - 1}]$ is the solution of the linear system.

\subsubsection{Natural cubic B-splines}
\label{sec:natural-cubic-b-splines}

For the natural cubic B-splines, we have $S''(t_1) = 0, S''(t_N) = 0$. With the same formula in \cref{eq:complete-cubic-b-spline}, it follows 
\begin{equation}
  a_{-1} = -\frac{3(t_2 - t_1)}{t_3 + t_2 - 2t_1} a_1,\quad a_N = -\frac{3(t_N - t_{N - 1})}{2t_N - t_{N - 1} - t_{N - 2}},
  \label{eq:natural-cubic-b-spline-coefficients}
\end{equation}
and the linear system $M\mathbf{a} = b$ with $\mathbf{a}^T = [a_0, a_1, \cdots, a_{N - 1}]$ is 
\begin{equation}
  \begin{aligned}
    \mathbf{b}^T &= [f(t_1), f(t_2), \cdots, f(t_{N - 1}), f(t_N)], \\
    M &= \begin{bmatrix}
      \frac{1}{3} + \frac{t_3 - t_1}{2(t_3 + t_2 - 2t_1)} &    &    &    &    \\
      B_0^3(t_2) & B_1^3(t_2) & B_2^3(t_2) &    &    \\
        & \ddots & \ddots & \ddots &    \\
        &    & B_{N - 3}^3(t_{N - 1}) & B_{N - 2}^3(t_{N - 1}) & B_{N - 1}^3(t_{N - 1}) \\
        &    &    &    & \frac{1}{3} + \frac{t_N - t_{N - 2}}{2(2t_N - t_{N - 1} - t_{N - 2})}
    \end{bmatrix}.
  \end{aligned}
  \label{eq:natural-cubic-b-spline-linear-system}
\end{equation}

\subsubsection{Periodic cubic B-splines}

In this section, we consider the periodic cubic B-splines. So we transfer \cref{eq:cubic-b-spline-knots} into 
\begin{equation}
  \begin{aligned}
    t_{-1} = t_1 + t_{N - 2} - t_N,\quad & t_0 = t_1 + t_{N - 1} - t_N, \\
    t_{N + 1} = t_N + t_2 - t_1,\quad & t_{N + 2} = t_N + t_3 - t_1.
  \end{aligned}
  \label{eq:periodic-cubic-b-spline-knots}
\end{equation}

We have the linear system $M\mathbf{a} = \mathbf{b}$ with $\mathbf{a}^T = [a_0, a_1, \cdots, a_{N - 1}]$ and
\begin{equation}
  \begin{aligned}
    \mathbf{b}^T &= [f(t_1), f(t_2), \cdots, f(t_{N - 1}), f(t_N)], \\
    M &= \begin{bmatrix}
      B_0^3(t_1) & B_1^3(t_1) &  \cdots  &  B_{-1}^3(t_1) &  \\
      B_0^3(t_2) & B_1^3(t_2) & B_2^3(t_2) &    &    \\
        &  \ddots &  \ddots & \ddots &    \\
        &    & B_{N - 3}^3(t_{N - 1}) & B_{N - 2}^3(t_{N - 1}) & B_{N - 1}^3(t_{N - 1}) \\
        & B_N^3(t_N) & \cdots &  B_{N - 2}^3(t_N) & B_{N - 1}^3(t_N) \\
    \end{bmatrix}.
  \end{aligned}
  \label{eq:periodic-cubic-b-spline-linear-system}
\end{equation}

\section{Method of solving linear systems}

In this section, we will discuss the method of solving linear systems. 

Obviously, the coefficient matrixes mentioned in the previous sections are sparse, so we need to implement the direct method to solve the linear systems. 

First we have the theorem 
\begin{theorem}
  Let $A$ be a nonsingular matrix. There exists a permutation matrix $P$, a lower triangular matrix $L$ with unit diagonal entries, and an upper triangular matrix $U$ such that 
  \begin{equation}
    PA = LU.
    \label{eq:lu-decomposition}
  \end{equation}
\end{theorem}

Once the LU decomposition $PA = LU$ is available, one can solve the linear system $Ax = b$ in the following way:
\begin{enumerate}
  \item Solve $Ly = Pb$ for $y$;
  \item Solve $Ux = y$ for $x$.
\end{enumerate}
Here the cost of solving the $n \times n$ triangular systems is $O(n^2)$. 

From the above analysis, we get the Gauss elimination with partial pivoting algorithm. The algorithm is as follows:
\begin{algorithm}[H]
  \For{$i = 1 : n - 1$}{
    permute rows so that $|a_{ii}|$ is the largest in $|A(i : n, i)|$ \;
    permute $L(i : n, 1 : i - 1)$ accordingly \;
    \For{$j = i + 1 : n$}{
      $l_{ji} \leftarrow a_{ji} / a_{ii}$ \;
    }
    \For{$j = i : n$}{
      $u_{ij} \leftarrow a_{ij}$ \;
    }
    \For{$j = i + 1 : n$}{
      \For{$k = i + 1 : n$}{
        $a_{jk} \leftarrow a_{jk} - l_{jk} u_{ik}$ \;
      }
    }
  }
  \caption{Gauss elimination with partial pivoting algorithm}
  \label{alg:gauss-elimination-with-partial-pivoting}
\end{algorithm}

From \cref{alg:gauss-elimination-with-partial-pivoting} we can see that the error of this algorithm will be influenced by the value of $a_{ii}$. For example, we consider the linear system $Ax = b$, where 
\begin{equation}
  A = \begin{pmatrix}
    \epsilon & 1 \\
    1 & 1
  \end{pmatrix}, \quad b = \begin{pmatrix}
    1 \\ 1
  \end{pmatrix}, \quad 0 < \epsilon \ll 1.
  \label{eq:linear-system-with-small-values}
\end{equation}
The exact solution is $x = [0, 1]^T$. Without machine round-off error, a standard Gaussian elimination implies 
\begin{equation}
  A = LU = \begin{pmatrix}
    1 & 0 \\ 
    -\frac{1}{\epsilon} & 1
  \end{pmatrix}
  \begin{pmatrix}
    -\epsilon & 1 \\
    0 & 1 + \frac{1}{\epsilon} 
  \end{pmatrix}.
  \label{eq:lu-decomposition-with-small-values}
\end{equation}

Considering the round-off process in the computer, we have 
\begin{equation}
  \hat{L} = \begin{pmatrix}
    1 & 0 \\ 
    -\frac{1}{\epsilon} & 1 
  \end{pmatrix}, \quad 
  \hat{U} = \begin{pmatrix}
    -\epsilon & 1 \\
    0 & \frac{1}{\epsilon} 
  \end{pmatrix}
  \label{eq:lu-decomposition-with-round-off}
\end{equation}
which yields that 
\begin{equation}
  \hat{A} = \begin{pmatrix}
    -\epsilon & 1 \\ 
    1 & 0
  \end{pmatrix} \ne \begin{pmatrix}
    \epsilon & 1 \\
    1 & 1
  \end{pmatrix}.
  \label{eq:linear-system-with-small-values-with-round-off}
\end{equation}

With the pivoting, the error of the algorithm will be obviously less than the machine round-off error.

\end{document}